\documentclass{article}
\usepackage{amssymb}
\usepackage{amsmath}
\usepackage{xifthen}
\usepackage[usenames,dvipsnames]{xcolor}
\usepackage{pgfplots, pgfplotstable}
\usepackage{etex}
\usepackage{tikz}
\usepackage{filecontents}
\usepackage{units}
\usetikzlibrary{
  bending,
	shadows,
	external,
	positioning,
	scopes,
	intersections,
	snakes,
	shapes.geometric,
	shapes,
	shapes.misc,
	shapes.gates.logic.IEC,
	shapes.gates.logic.US,
	arrows,
	backgrounds,
	automata,
	3d,
	calc,
	chains,
	er,
	matrix,
	fit,
	fadings,
	patterns,
	calendar,
	decorations,
	decorations.pathmorphing,
	decorations.pathreplacing,
  decorations.markings,
	circuits,
	circuits.ee.IEC,
	through,
	topaths
} %% SOBRAN MUCHAS

\tikzexternalize[mode=list and make,prefix=./]
\tikzset{external/system call={pdflatex \tikzexternalcheckshellescape -shell-escape -halt-on-error -interaction=batchmode -jobname "\image" "\texsource" && pdftops -eps "\image".pdf "\image".foo.eps && epstool --copy --bbox "\image".foo.eps "\image".eps && rm "\image".foo.eps "\image".pdf "\image".dpth "\image".log "\image".dep}}

%MACRO of the different elements for block diagram
\begin{document}
	% Libraries
	% neuralNetwork
	\tikzsetnextfilename{neuralNetwork}% name next TikZ figure
	\def\layersep{1.5cm}
\def\numberNeuronInput{1}
\def\numberNeuronHidden{1}
\def\numberNeuronOutput{1}
\begin{tikzpicture}[shorten >=1pt,draw=black!50, node distance=\layersep]
  \tikzset{normal arrow/.style={draw,-{stealth[black!50]},shorten <=0pt}}
	\tikzset{body arrow/.style={draw,shorten <=0pt, shorten >= 0pt}}
  \tikzset{head arrow/.style={draw,-{stealth[black!50]},shorten <=0pt, shorten >= 0pt}}
  \tikzstyle{every pin edge}=[{stealth[black!50]}-,shorten <=1pt]
  \tikzstyle{neuron}=[circle,draw=black!90,fill=black!25,minimum size=17pt,inner sep=0pt]
  \tikzstyle{input neuron}=[neuron, fill=red!50];
  \tikzstyle{output neuron}=[neuron, fill=violet!50];
  \tikzstyle{hidden neuron}=[neuron, fill=black!20!yellow!75];
  \tikzstyle{annot} = [text width=4em, text centered]
  
  \node[input neuron, pin=left:$y(t-\Delta t)$] (I-1) at (0,-1) {1};
  \node[input neuron] (I-2) at (0,-2) {2};

  % Draw the hidden layer nodes
  \foreach \name / \y in {1,...,3}
    \pgfmathsetmacro\result{\name + 2}
    \path[yshift=0.5cm]
      node[hidden neuron] (H-\name) at (\layersep,-\y cm) {\pgfmathprintnumber{\result}};

  % Draw the output layer node
  \node[output neuron,pin={[pin edge={normal arrow}]right:$x_i(t)$}, right of=H-2] (O-1) {6};

  % Connect every node in the input layer with every node in the
  % hidden layer.
  \foreach \source in {1,...,\numberNeuronInput}
    \foreach \dest in {1,...,3}
      \draw[normal arrow] (I-\source.east) -- (H-\dest);

  % Connect every node in the hidden layer with the output layer
  \foreach \source in {1,...,\numberNeuronHidden}
    \draw[normal arrow] (H-\source.east) -- (O-1);

  %Self feedback
  %Input
  \foreach \x in {1,...,\numberNeuronInput} {
    \coordinate (C1) at ($(I-\x.east) + (0.005,0)$);
    \coordinate (C2) at ($(I-\x.north) + (0.35,0.15)$);
	  \coordinate (C3) at ($(I-\x.north) + (-0.055,0.17)$);
	  \coordinate (C4) at ($(I-\x.north) + (-0.1,0)$);
    \draw [head arrow] (C1) to[out=45,in=-45] (C2) to[out=135,in=45] (C3) to[out=-135,in=90] (C4);
  }
  %Hidden
  \foreach \x in {1,...,\numberNeuronHidden} {
    \coordinate (C1) at ($(H-\x.east) + (0,0)$);
    \coordinate (C2) at ($(H-\x.north) + (0.35,0.15)$);
	  \coordinate (C3) at ($(H-\x.north) + (-0.055,0.17)$);
	  \coordinate (C4) at ($(H-\x.north) + (-0.1,0)$);
    \draw [head arrow] (C1) to[out=45,in=-45] (C2) to[out=135,in=45] (C3) to[out=-135,in=90] (C4);
  }
  \draw[normal arrow] (H-1.east) to[out=45,in=-45] ($(H-1.north) + (0.35,0.15)$) to[out=135,in=45] ($(H-1.north) + (-0.25,0.1)$) to[out=-135,in=135] ($(H-2.west) + (0.1,0.25)$);
  \draw[normal arrow] (H-1.east) to[out=45,in=-45] ($(H-1.north) + (0.35,0.15)$) to[out=135,in=45] ($(H-1.north) + (-0.25,0.1)$) to[out=-135,in=135] ($(H-3.north) + (-0.05,0.025)$);
  %Output
  \foreach \x in {1,...,\numberNeuronOutput} {
    \coordinate (C1) at ($(O-\x.east) + (0.015,0)$);
    \coordinate (C2) at ($(O-\x.north) + (0.35,0.15)$);
	  \coordinate (C3) at ($(O-\x.north) + (-0.055,0.17)$);
	  \coordinate (C4) at ($(O-\x.north) + (-0.1,0)$);
    \draw [head arrow] (C1) to[out=45,in=-45] (C2) to[out=135,in=45] (C3) to[out=-135,in=90] (C4);
  }
  %Feedback for outpu to the input
  \coordinate (C1) at ($(O-1.east) + (0.015,0)$);
  \coordinate (C2) at ($(I-2.south) + (0,-0.75)$);
  \coordinate (C3) at ($(I-2.west) + (-0.4123,0.03)$);
  \draw[normal arrow] ($(O-1.east) + (0.015,0)$) to[out=-45,in=90] ($(O-1.east) + (0.4,-0.75)$) to[out=-90,in=0] ($(I-2.south) + (0.75,-1.1)$) to[out=180,in=-90] ($(I-2.west) + (-0.4,-0.4)$) to[out=90,in=180] ($(I-2.west) + (0,0)$);
  \draw[normal arrow] ($(O-1.east) + (0.015,0)$) to[out=-45,in=90] ($(O-1.east) + (0.4,-0.75)$) to[out=-90,in=0] ($(I-2.south) + (0.75,-1.1)$) to[out=180,in=-90] ($(I-2.west) + (-0.4,-0.4)$) to[out=90,in=-135] ($(I-1.west) + (0,-0.15)$);
  %Other two inputs
  \draw[normal arrow] ($(I-1.west) + (-0.4123,0)$) -- ($(I-2.west) + (-0.01,-0.01)$);
\end{tikzpicture}
	 % input TiKZ figure code
\end{document}
